\chapter{Introduction}
\label{cp:introduction}
The first part of this lab was to calibrate a pressure sensor and use the data to quantify uncertainty of measurements collected. This part is done using a plenum chamber to control pressure and a Mensor manometer to act as the standard for the pressure sensor. The second part of the lab used the calibrated pressure sensor data to create the q-map (The dynamic pressure distribution) inside the test section of the wind tunnel. A pitot tube is placed at different points in the test section at a constant wind tunnel motor frequency to measure the dynamic pressure.\url{https://www.aere.iastate.edu/~huhui/teaching/2024-01S/AerE344/lab-instruction/AerE344L-Lab-03-instruction.pdf}