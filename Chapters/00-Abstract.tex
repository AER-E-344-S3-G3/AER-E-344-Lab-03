\thispagestyle{plain} % Page style without header and footer
% \pdfbookmark[1]{Resumo}{resumo} % Add entry to PDF
% \chapter*{Resumo} % Chapter* to appear without numeration
% \blindtext

% \keywordspt{Keyword A, Keyword B, Keyword C.}

% \blankpage

\pdfbookmark[1]{Abstract}{abstract} % Add entry to PDF
\chapter*{Abstract} % Chapter* to appear without numeration
To measure the dynamic pressure and velocity distribution of the low-speed wind tunnel at Iowa State University, we first had to calibrate the pressure transducer we would use to measure the pitot tube pressures in the wind tunnel. By using a plenum chamber and a reference pressure gauge, we calculated the calibration coefficient of our Omega pressure transducer to be $C = \qty{623.3}{\pascal\per\volt}$. We then used this calibration coefficient and the Omega pressure transducer to measure the dynamic pressure at distances of \qtyrange{0}{16}{\centi\meter} from the wall of the wind tunnel test chamber. This dynamic pressure data was used to calculate and plot a velocity distribution of the low-speed wind tunnel, which can be used to determine how far objects should be from the wall of the test chamber to avoid boundary layer effects.

% \blankpage


