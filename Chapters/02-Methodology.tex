\chapter{Methodology}
\label{cp:methodology}

\textbf{Pressure Sensor Calibration:} \\
\underline{Apparatus:} Tubes are connected from the Mensor Digital Pressure Gauge to the plenum chamber and from the Setra electronic manometer to the plenum chamber. The pump is connected to the valve on the plenum chamber. \\
\underline{Procedure:} 
\begin{enumerate}
    \item Measure the voltage with 0 pressure in the chamber. 
    \item Pressurize close to 5 in H20 and record the voltage. 
    \item Open the chamber valve and release about 0.5 in H20 of pressure.  
    \item Wait for the pressure to stabilize and record the voltage.  
    \item Repeat steps 3-4 until the pressure is 0. Take at least 10 data points. If you reach 0 pressure before you have 10 data points, re-pressurize to 5 inH20 and repeat the process until you have acquired enough data.
\end{enumerate}
\noindent\textbf{Measurement of the Undergraduate Aerospace Wind Tunnel}\\
\underline{Apparatus:} The pitot tube is connected to the Setra electronic manometer. The total pressure tube of the test section is connected to the main port on the electronic manometer and the static pressure tube is connected to the reference port of the manometer. \\
\underline{Procedure:}
\begin{enumerate}
    \item Set the wind tunnel to 10 m/s. 
    \item Start the pitot tube at the wall and sample the voltage from the electronic manometer. Save the file. 
    \item Move the pitot tube in from the wall 1 cm. and sample the voltage from the electronic manometer. Save the file.  
    \item Repeat for step 3 for 15 locations, until the pitot tube should be in the middle of the test section.
\end{enumerate}