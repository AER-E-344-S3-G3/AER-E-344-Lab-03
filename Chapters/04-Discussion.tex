\chapter{Discussion}
\label{cp:discussion}
Analyzing the relationship between pressure and voltage in <figure graph 1>, we can see that it is linear. From this, we can assume that there is a direct relationship between voltage and pressure. Using the polyfit function in MatLab, we were able to derive <equation of line 1> and confirm a strong direct relationship, due to the the coefficient of determination. $R^2$, = <insert $R^2$ value>. From this equation, we were able to find the calibration constant, C. We found C = 622.6634. This allows us to form an equation to model the relationship between pressure and the voltage from the manometer.  

The dynamic pressure is used to calculate the velocity at each point from the wall using $q =\frac{1}{2}\rho v^2$. Since we assume a constant air density throughout the test section, the graphs of velocity and dynamic pressure vs the distance are a scalar of each other. There is a nonlinear relationship between the dynamic pressure and velocity vs the distance of the pitot tube from the wall in the test section. The relationship looks logarithmic as values of dynamic pressure and velocity are drastically lower very close to the wall. This due to friction between the air and the wind tunnel walls slowing the air down.