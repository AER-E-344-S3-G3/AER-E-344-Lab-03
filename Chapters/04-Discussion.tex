\chapter{Discussion}
\label{cp:discussion}

\autoref{fig:pressure_vs_voltage} shows the relationship between known pressure values and the voltage output of the Omega pressure transducer. The coefficient of determination, \gls{Rsquared}, was extremely high: $R^2 = 0.999997$. This extremely precise correlation is due to the accuracy and precision of the Omega pressure transducer. This \gls{Rsquared} value \textit{does not} mean our calibration coefficient, \gls{C}, is accurate to the same degree. Since the pressure values were logged manually and the Mensor digital pressure gauge would not consistently settle to a certain pressure, the calibration constant may be inaccurate. See \autoref{cp:C_error} for a discussion on the calibration constant error.

\autoref{fig:pressure_vs_voltage} and \autoref{fig:velocity_vs_distance} both show the general shape and trends we expected. 

% The dynamic pressure is used to calculate the velocity at each point from the wall using \autoref{eq:dynamic_pressure}. Since we assume a constant air density throughout the test section, the graphs of velocity and dynamic pressure vs the distance are a scalar of each other. There is a nonlinear relationship between the dynamic pressure and velocity vs the distance of the pitot tube from the wall in the test section. The relationship looks logarithmic as values of dynamic pressure and velocity are drastically lower very close to the wall. This is due to friction between the air and the wind tunnel walls slowing the air down.