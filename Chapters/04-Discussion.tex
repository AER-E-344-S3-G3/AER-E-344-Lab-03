\chapter{Discussion}
\label{cp:discussion}

\autoref{fig:pressure_vs_voltage} shows the relationship between known pressure values and the voltage output of the Omega pressure transducer. The coefficient of determination, \gls{Rsquared}, was extremely high: $R^2 = 0.999997$. This extremely precise correlation is due to the accuracy and precision of the Omega pressure transducer. This \gls{Rsquared} value \textit{does not} mean our calibration coefficient, \gls{C}, is accurate to the same degree. Since the pressure values were logged manually and the Mensor digital pressure gauge would not consistently settle to a certain pressure, the calibration constant may be inaccurate. See \autoref{cp:C_error} for a discussion on the calibration constant error.

\autoref{fig:pressure_vs_voltage} and \autoref{fig:velocity_vs_distance} both show the general shape and trends we expected. At the wall of the wind tunnel, $L=0$, the flow velocity is significantly lower than the rest of the test chamber due to viscous boundary layer effects. Further from the wall, the velocity levels out to the expected flow speed. By using the wind tunnel calibration constant, \gls{K}, that we calculated in lab two, we determined that the air speed in the wind tunnel for a \qty{15}{\hertz} motor frequency should be approximately \qty{19.4}{\meter\per\second}. \autoref{fig:velocity_vs_distance} shows the velocity in the wind tunnel near the center of the test chamber was close to the expected velocity. The approximate \qty{1}{\meter\per\second} difference is likely due to inaccuracy in our pressure transducer calibration coefficient, \gls{C}, as discussed in \autoref{cp:C_error}.

This data is useful to determine the useful width of the wind tunnel if an object with finite width is being measured. Examining \autoref{fig:dynamic_pressure_vs_distance}, it is clear that any object or probes placed within \qty{6}{\centi\meter} of the test chamber walls will experience unrepresentative flow conditions compared to the center of the test chamber.

% The dynamic pressure is used to calculate the velocity at each point from the wall using \autoref{eq:dynamic_pressure}. Since we assume a constant air density throughout the test section, the graphs of velocity and dynamic pressure vs the distance are a scalar of each other. There is a nonlinear relationship between the dynamic pressure and velocity vs the distance of the pitot tube from the wall in the test section. The relationship looks logarithmic as values of dynamic pressure and velocity are drastically lower very close to the wall. This is due to friction between the air and the wind tunnel walls slowing the air down.